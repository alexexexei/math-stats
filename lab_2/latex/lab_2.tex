\documentclass[a4paper, 12pt]{article}
\usepackage[utf8x]{inputenc}
\usepackage{cmap}
\usepackage[english, russian]{babel}
\usepackage{indentfirst}
\usepackage[left=20mm, top=20mm, right=20mm, bottom=20mm]{geometry}
\usepackage{tikz}
\usepackage{float}
\usepackage{amsmath, amsfonts, amssymb}
\usepackage{graphicx}
\usepackage{fancybox, fancyhdr}
\usepackage{hyperref}
\usepackage{listings}
\usepackage{caption}
\usepackage{subcaption}
\usepackage{xcolor}
\pagestyle{fancy}
\fancyhf{}
\fancyhead[L]{Лабораторная работа №2}
\fancyhead[R]{Математическая статистика}
\fancyfoot[C]{\thepage}
\graphicspath{{images/}}
\usetikzlibrary{patterns}
\definecolor{LightGray}{gray}{0.95}
\definecolor{LightGray2}{gray}{0.7}
\lstdefinestyle{code}{
    language=Python, % replace with needed language
    basicstyle=\footnotesize\ttfamily,
    backgroundcolor=\color{LightGray},
    showspaces=false,
    showstringspaces=false,
    showtabs=false,
    tabsize=4,
    captionpos=b,
    breaklines=true,
    breakatwhitespace=false,
    frame=single,
    rulecolor=\color{LightGray2},
    linewidth=\linewidth,
    keywordstyle=\color{blue}\bfseries,
    commentstyle=\color{green!40!black},
    stringstyle=\color{purple},
    escapeinside={\%*}{*)},
    inputencoding=utf8x,
    xleftmargin=0pt,
    framexleftmargin=0pt,
    framexrightmargin=0pt
}
\lstset{style=code}
\hypersetup{
    colorlinks=true,
    linkcolor=blue,
    filecolor=magenta,
    urlcolor=cyan,
    pdftitle={contents setup},
    pdfpagemode=FullScreen,
}
\setlength{\parskip}{1.5mm}
\setlength{\headheight}{15pt}
\setlength{\footskip}{15pt}
\allowdisplaybreaks
\DeclareMathOperator{\sinc}{sinc}
\newcommand{\frc}[2]{\raisebox{2pt}{$#1$}\big/\raisebox{-3pt}{$#2$}}

\begin{document}
    \begin{titlepage}

        \begin{center}
        \includegraphics[width=0.3\textwidth]{itmo.png} % requires itmo.png in /images folder
        \vfill
        
        Федеральное государственное автономное образовательное учреждение высшего образования
        «Национальный Исследовательский Университет ИТМО»\\
        
        \vfill
        {\large\bf ЛАБОРАТОРНАЯ РАБОТА №2}\\
        {\large\bf ПРЕДМЕТ «МАТЕМАТИЧЕСКАЯ СТАТИСТИКА»}\\
        {\large\bf ТЕМА «ДОВЕРИТЕЛЬНЫЕ ИНТЕРВАЛЫ»}\\
        Вариант 1, 1
        \vfill

        \begin{flushright}
            \begin{minipage}{.45\textwidth}
            {
                \hbox{Преподаватель: Лимар И. А.}
                \hbox{Студент: Румянцев А. А.}
                \hbox{Поток: Мат Стат 31.2}
                \hbox{}
                \hbox{Факультет: СУиР}
                \hbox{Группа: R3341}
            }
            \end{minipage}
        \end{flushright}
        
        \vfill
                
        Санкт-Петербург\\
        2024
        \end{center}
    \end{titlepage}
    
    \tableofcontents

    \newpage
    \section{Задание 1}
    \subsection{Условие}
    Предъявите доверительный интервал уровня $1-\alpha$ для указанного параметра при данных
    предположениях (с математическими обоснованиями). Сгенерируйте 2 выборки объёма объёма
    25 и посчитайте доверительный интервал. Повторить 1000 раз. Посчитайте, сколько раз 
    95-процентный доверительный интервал покрывает реальное значение параметра. То же самое
    сделайте для объема выборки 10000. Как изменился результат? Как объяснить? Что изменяется
    при росте объемов выборок?


    Даны две независимые выборки $X_1$, $X_2$ из нормальных распределений $\mathcal{N}\left(\mu_1,\sigma_1^2\right)$,\\
    $\mathcal{N}(\mu_2,\sigma_2^2)$ объемов $n_1$, $n_2$ соответственно.
    Сначала указывается оцениваемая функция, потом данные об остальных параметрах, затем параметры эксперимента и подсказки.
    $$\tau=\mu_1-\mu_2;\,\sigma_1^2,\,\sigma_2^2\text{ известны};\,\mu_1=2,\,\mu_2=1,\,\sigma_1^2=1,\,\sigma_2^2=0.5;\text{ воспользуйтесь функцией}$$
    $$\dfrac{\overline{X_1}-\overline{X_2}-\tau}{\sigma},\,\,\,\,\sigma^2=\dfrac{\sigma_1^2}{n_1}+\dfrac{\sigma_2^2}{n_2}$$


    \subsection{Выполнение}
    тут будет выполнение


    \section{Задание 2}
    \subsection{Условие}
    Постройте асимптотический доверительный интервал уровня $1-\alpha$ для указанного параметра.
    Проведите эксперимент по схеме, аналогичной первой задаче.
    
    
    Сначала указывается класс распределений (однопараметрический),
    затем параметры эксперимента и подсказки.
    $$\text{Ехр}(\lambda);\text{ медиана};\,\lambda=1$$


    \subsection{Выполнение}
    тут будет выполнение
\end{document}