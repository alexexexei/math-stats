\documentclass[a4paper, 12pt]{article}
\usepackage[utf8x]{inputenc}
\usepackage{cmap}
\usepackage[english, russian]{babel}
\usepackage{indentfirst}
\usepackage[left=20mm, top=20mm, right=20mm, bottom=20mm]{geometry}
\usepackage{tikz}
\usepackage{float}
\usepackage{amsmath, amsfonts, amssymb}
\usepackage{graphicx}
\usepackage{fancybox, fancyhdr}
\usepackage{hyperref}
\usepackage{listings}
\usepackage{caption}
\usepackage{subcaption}
\usepackage{xcolor}
\usepackage{attachfile2}
\pagestyle{fancy}
\fancyhf{}
\fancyhead[L]{Лабораторная работа №2}
\fancyhead[R]{Математическая статистика}
\fancyfoot[C]{\thepage}
\graphicspath{{images/}}
\usetikzlibrary{patterns}
\definecolor{LightGray}{gray}{0.95}
\definecolor{LightGray2}{gray}{0.7}
\lstdefinestyle{code}{
    language=Python, % replace with needed language
    basicstyle=\footnotesize\ttfamily,
    backgroundcolor=\color{LightGray},
    showspaces=false,
    showstringspaces=false,
    showtabs=false,
    tabsize=4,
    captionpos=b,
    breaklines=true,
    breakatwhitespace=false,
    frame=single,
    rulecolor=\color{LightGray2},
    linewidth=\linewidth,
    keywordstyle=\color{blue}\bfseries,
    commentstyle=\color{green!40!black},
    stringstyle=\color{purple},
    escapeinside={\%*}{*)},
    inputencoding=utf8x,
    xleftmargin=0pt,
    framexleftmargin=0pt,
    framexrightmargin=0pt
}
\lstset{style=code}
\hypersetup{
    colorlinks=true,
    linkcolor=blue,
    filecolor=magenta,
    urlcolor=cyan,
    pdftitle={contents setup},
    pdfpagemode=FullScreen,
}
\setlength{\parskip}{1.5mm}
\setlength{\headheight}{15pt}
\setlength{\footskip}{15pt}
\allowdisplaybreaks
\DeclareMathOperator{\sinc}{sinc}
\newcommand{\frc}[2]{\raisebox{2pt}{$#1$}\big/\raisebox{-3pt}{$#2$}}

\begin{document}
    \begin{titlepage}

        \begin{center}
        \includegraphics[width=0.3\textwidth]{itmo.png} % requires itmo.png in /images folder
        \vfill
        
        Федеральное государственное автономное образовательное учреждение высшего образования
        «Национальный Исследовательский Университет ИТМО»\\
        
        \vfill
        {\large\bf ЛАБОРАТОРНАЯ РАБОТА №2}\\
        {\large\bf ПРЕДМЕТ «МАТЕМАТИЧЕСКАЯ СТАТИСТИКА»}\\
        {\large\bf ТЕМА «ДОВЕРИТЕЛЬНЫЕ ИНТЕРВАЛЫ»}\\
        Вариант 1, 1
        \vfill

        \begin{flushright}
            \begin{minipage}{.45\textwidth}
            {
                \hbox{Преподаватель: Лимар И. А.}
                \hbox{Студент: Румянцев А. А.}
                \hbox{Поток: Мат Стат 31.2}
                \hbox{}
                \hbox{Факультет: СУиР}
                \hbox{Группа: R3341}
            }
            \end{minipage}
        \end{flushright}
        
        \vfill
                
        Санкт-Петербург\\
        2024
        \end{center}
    \end{titlepage}
    
    \tableofcontents

    \newpage
    \section{Задание 1}
    \subsection{Условие}
    Предъявите доверительный интервал уровня $1-\alpha$ для указанного параметра при данных
    предположениях (с математическими обоснованиями). Сгенерируйте 2 выборки объёма объёма
    25 и посчитайте доверительный интервал. Повторить 1000 раз. Посчитайте, сколько раз 
    95-процентный доверительный интервал покрывает реальное значение параметра. То же самое
    сделайте для объема выборки 10000. Как изменился результат? Как объяснить? Что изменяется
    при росте объемов выборок?


    Даны две независимые выборки $X_1$, $X_2$ из нормальных распределений $\mathcal{N}\left(\mu_1,\sigma_1^2\right)$,\\
    $\mathcal{N}(\mu_2,\sigma_2^2)$ объемов $n_1$, $n_2$ соответственно.
    Сначала указывается оцениваемая функция, потом данные об остальных параметрах, затем параметры эксперимента и подсказки.
    $$\tau=\mu_1-\mu_2;\,\sigma_1^2,\,\sigma_2^2\text{ известны};\,\mu_1=2,\,\mu_2=1,\,\sigma_1^2=1,\,\sigma_2^2=0.5;\text{ воспользуйтесь функцией}$$
    $$\dfrac{\overline{X_1}-\overline{X_2}-\tau}{\sigma},\,\,\,\,\sigma^2=\dfrac{\sigma_1^2}{n_1}+\dfrac{\sigma_2^2}{n_2}$$


    \subsection{Выполнение}
    Пусть $X_1,\hdots,X_n\sim\mathcal{N}\left(\mu,\sigma^2\right)$ -- независимая выборка из нормального распределения,
    $\overline{X}$ -- выборочное среднее. Тогда, по теореме Фишера, среднее выборочное также имеет нормальное распределение
    $$Z=\sqrt{n}\cdot\dfrac{\overline{X}-\mu}{\sigma}\sim\mathcal{N}\left(0,1\right).$$
    По условию задания имеем неизвестный параметр $\mu$ -- математическое ожидание (генеральное среднее) случайной величины $X$. В качестве точечной
    оценки параметра $\mu$ возьмем выборочное среднее $\hat{\mu}=\overline{X}$. Для уточнения приближенного равенства
    $\mu\approx \overline{X}$ построим доверительный интервал, накрывающий параметр $\mu$ с заданной доверительной
    вероятностью $$\gamma=1-\alpha,$$ при этом статистика $$\overline{X}=\dfrac{1}{n}\sum\limits_{i=1}^{n}x_i$$
    имеет нормальное распределение с параметрами $$\overline{X}\sim\mathcal{N}\left(\mu,\dfrac{\sigma^2}{n}\right).$$
    Обозначим границы интервалов через квантили порядков $\frac{\alpha}{2}$ и $1-\frac{\alpha}{2}$ соответственно
    $$r_1=x_{\frac{\alpha}{2}},\,\,r_2=x_{1-\frac{\alpha}{2}},$$ где $r_1$ -- нижняя граница доверительного интервала, $r_2$ -- верхняя.
    Тогда, доверительная вероятность $\gamma$ удовлетворяет соотношению
    $$\mathbb{P}\left(r_1\leq\sqrt{n}\cdot\dfrac{\overline{X}-\mu}{\sigma}\leq r_2\right)=\gamma=1-\alpha,$$
    что соответствует рис. \ref{fig:normal}
    \begin{figure}[H]
        \centering
        \includegraphics[scale=1.2]{normal.png}
        \captionsetup{skip=0pt}
        \caption{Двусторонняя критическая область}
        \label{fig:normal}
    \end{figure}
    \noindent Пользуясь свойством симметричности нормального распределения
    $$r_1=-r_2=-x_{1-\frac{\alpha}{2}}.$$
    Таким образом, исходя из приведенного ранее соотношения и симметричности границ интервала, необходимо выразить интервал для $\mu$ из выражения 
    $$\Bigg|\sqrt{n}\cdot\dfrac{\overline{X}-\mu}{\sigma}\Bigg|\leq x_{1-\frac{\alpha}{2}}$$


    В нашем случае имеем две выборки из нормальных распределений. Исходя из предыдущих рассуждений, обе эти выборки
    также будут иметь нормальные распределения с параметрами
    $$\overline{X_1}\sim\mathcal{N}\left(\mu_1,\dfrac{\sigma_1^2}{n_1}\right),\,\,\,\,\overline{X_2}\sim\mathcal{N}\left(\mu_2,\dfrac{\sigma_2^2}{n_2}\right)$$


    Так как выборочное среднее и математическое ожидание обладают свойством линейности, то разность выборок с нормальными распределениями
    даст выборку с нормальным распределением с параметрами
    $$\overline{X_1}-\overline{X_2}\sim\mathcal{N}\left(\mu_1-\mu_2,\dfrac{\sigma_1^2}{n_1}+\dfrac{\sigma_2^2}{n_2}\right),$$
    что с учетом наших замен можно записать как $$\overline{X_1}-\overline{X_2}\sim\mathcal{N}\left(\tau,\sigma^2\right).$$
    Преобразуем к такому виду, чтобы справа осталось стандартное нормальное распределение $\mathcal{N}\left(0,1\right).$
    Вычтем из левой и правой части $\tau$
    $$\overline{X_1}-\overline{X_2}-\tau\sim\mathcal{N}\left(0,\sigma^2\right),$$
    теперь поделим правую часть на $\sigma^2$, а левую на $\sqrt{\sigma^2}$
    $$\dfrac{\overline{X_1}-\overline{X_2}-\tau}{\sigma}\sim\mathcal{N}\left(0,1\right).$$
    Таким образом слева получили данное в условии задания выражение, которое имеет стандартное нормальное распределение.
    Преобразуем дисперсию генеральной совокупности так, чтобы при подстановке в выражение выше получить искомое выражение
    $$\sigma=\sqrt{\dfrac{\sigma_1^2}{n_1}+\dfrac{\sigma_2^2}{n_2}}=\sqrt{\dfrac{n_2\cdot\sigma_1^2+n_1\cdot\sigma_2^2}{n_1\cdot n_2}}=\dfrac{\sqrt{n_2\cdot\sigma_1^2+n_1\cdot\sigma_2^2}}{\sqrt{n_1\cdot n_2}}.$$
    Подставим преобразованную дисперсию в данное в задаче выражение
    $$\dfrac{\overline{X_1}-\overline{X_2}-\tau}{\dfrac{\sqrt{n_2\cdot\sigma_1^2+n_1\cdot\sigma_2^2}}{\sqrt{n_1\cdot n_2}}}=\sqrt{n_1\cdot n_2}\cdot\dfrac{\overline{X_1}-\overline{X_2}-\tau}{\sqrt{n_2\cdot\sigma_1^2+n_1\cdot\sigma_2^2}}$$
    Мы получили выражение, похожее на искомое. Теперь определеим доверительный интервал для $\tau$. Выражения под корнями будут всегда положительны, так как в любой выборке должен быть хотя бы один элемент,
    дисперсия в четной степени, и, сумма положительных значений не может быть отрицательной. Следовательно, вынесем их из под модуля
    $$\Bigg|\sqrt{n_1\cdot n_2}\cdot\dfrac{\overline{X_1}-\overline{X_2}-\tau}{\sqrt{n_2\cdot\sigma_1^2+n_1\cdot\sigma_2^2}}\Bigg|\leq x_{1-\frac{\alpha}{2}}\Rightarrow
    \Bigg|\overline{X_1}-\overline{X_2}-\tau\Bigg|\leq x_{1-\frac{\alpha}{2}}\cdot\dfrac{\sqrt{n_2\cdot\sigma_1^2+n_1\cdot\sigma_2^2}}{\sqrt{n_1\cdot n_2}}$$
    Раскроем модуль и преобразуем неравенство так, чтобы в его рамках осталась только $\tau$
    $$-x_{1-\frac{\alpha}{2}}\cdot\dfrac{\sqrt{n_2\cdot\sigma_1^2+n_1\cdot\sigma_2^2}}{\sqrt{n_1\cdot n_2}}\leq\overline{X_1}-\overline{X_2}-\tau\leq x_{1-\frac{\alpha}{2}}\cdot\dfrac{\sqrt{n_2\cdot\sigma_1^2+n_1\cdot\sigma_2^2}}{\sqrt{n_1\cdot n_2}},$$
    $$-\left(\overline{X_1}-\overline{X_2}\right)-x_{1-\frac{\alpha}{2}}\cdot\dfrac{\sqrt{n_2\cdot\sigma_1^2+n_1\cdot\sigma_2^2}}{\sqrt{n_1\cdot n_2}}\leq-\tau\leq -\left(\overline{X_1}-\overline{X_2}\right)+x_{1-\frac{\alpha}{2}}\cdot\dfrac{\sqrt{n_2\cdot\sigma_1^2+n_1\cdot\sigma_2^2}}{\sqrt{n_1\cdot n_2}},$$
    $$\left(\overline{X_1}-\overline{X_2}\right)-x_{1-\frac{\alpha}{2}}\cdot\dfrac{\sqrt{n_2\cdot\sigma_1^2+n_1\cdot\sigma_2^2}}{\sqrt{n_1\cdot n_2}}\leq\tau\leq \left(\overline{X_1}-\overline{X_2}\right)+x_{1-\frac{\alpha}{2}}\cdot\dfrac{\sqrt{n_2\cdot\sigma_1^2+n_1\cdot\sigma_2^2}}{\sqrt{n_1\cdot n_2}}$$
    Теперь свернем дисперсию к изначальному виду и запишем полученный доверительный интервал
    $$\left(\overline{X_1}-\overline{X_2}\right)-x_{1-\frac{\alpha}{2}}\cdot\sigma\leq\tau\leq \left(\overline{X_1}-\overline{X_2}\right)+x_{1-\frac{\alpha}{2}}\cdot\sigma$$


    Проведем эксперимент. Для начала импортируем необходимые библиотеки
    \begin{lstlisting}[label=imps1, caption={Импортирование необходимых библиотек}]
    import numpy as np
    import scipy.stats as st
    \end{lstlisting}
    Запишем в переменные данные в условии значения параметров. Зададим стандартный уровень значимости $\alpha=5\%=0.05$
    \begin{lstlisting}[label=vars, caption={Задаем данные по условию}]
    mu_1, mu_2 = 2, 1
    tau = mu_1 - mu_2
    sigma2_1, sigma2_2 = 1, 0.5
    n_1, n_2 = 25, 25
    alpha = 0.05
    it = 1000
    \end{lstlisting}
    Реализуем основной алгоритм -- в цикле итерируемся $it$ раз и считаем $it$ доверительных интервалов для объемов $n_1,n_2$ по выведенной формуле.
    Проверим, попадает ли реальное значение параметра в доверительный интервал. Если True, то увеличим счетчик на 1. В конце
    выведем количество попавших в интервал $\tau$ и отношение относительно общего количества итераций
    \begin{lstlisting}[label=code1, caption={Код для подсчета доверительных интервалов и кол-ва попаданий}]
    count = 0
    for i in range(it):
        X_1 = np.random.normal(mu_1, sigma2_1, n_1)
        X_2 = np.random.normal(mu_2, sigma2_2, n_2)
        
        X_1_mean = np.mean(X_1)
        X_2_mean = np.mean(X_2)
        
        sigma = np.sqrt(sigma2_1 / n_1 + sigma2_2 / n_2)
        
        z = st.norm.ppf(1 - alpha / 2)
        
        lower_bound = (X_1_mean - X_2_mean) - z * sigma
        upper_bound = (X_1_mean - X_2_mean) + z * sigma

        print(f'{lower_bound:.4f}<={tau}<={upper_bound:.4f}')
            
        if lower_bound <= tau <= upper_bound:
            count += 1
        
    print(f'covers_tau_count={count}, ratio={count / it}')
    \end{lstlisting}


    Выведем доверительный интервал для $n=25,\,it=1$
    \begin{lstlisting}[label=res1, caption={Посчитанный доверительный интервал}]
    0.5934<=1<=1.5536
    \end{lstlisting}
    Выведем количество попавших в интервал $\tau$ и отношение относительно общего количества итераций
    для $n=25,\,it=1000$. Сами доверительные интервалы можно посмотреть в приложении 1
    \begin{lstlisting}[label=n25, caption={95-\% доверительный интервал для $n=25$}]
    covers_tau_count=970, ratio=0.97
    \end{lstlisting}
    Сделаем то же самое для $n=10000$
    \begin{lstlisting}[label=n10000, caption={95-\% доверительный интервал для $n=10000$}]
    covers_tau_count=960, ratio=0.96
    \end{lstlisting}


    При увеличении объема выборки отношение количества попавших в доверительный интервал $\tau$ к общему
    количеству проверок на попадание в доверительный интервал стремится к уровню доверия $\gamma=1-\alpha=1-0.05=0.95$.
    Доверительные интервалы становятся точнее, так как стандартная ошибка уменьшается. Однако большой
    объем выборки не гарантирует результат ровно в $95\%$ -- из-за вариации в выборках значение будет
    чаще всего либо больше, либо меньше $0.95$.


    \section{Задание 2}
    \subsection{Условие}
    Постройте асимптотический доверительный интервал уровня $1-\alpha$ для указанного параметра.
    Проведите эксперимент по схеме, аналогичной первой задаче.
    
    
    Сначала указывается класс распределений (однопараметрический),
    затем параметры эксперимента и подсказки.
    $$\text{Ехр}(\lambda);\text{ медиана};\,\lambda=1$$


    \subsection{Выполнение}
    Экспоненциальное распределение задается следующим образом
    $$f_X(x)=
    \begin{cases}
        \lambda\cdot\exp{\left\{-\lambda x\right\}},& x\geq0\\
        0,& x < 0
    \end{cases}$$
    Пусть есть некоторое распределение с неизвестным параметром $\theta$.
    Известно, что при увеличении объема выборки оценка этого параметра асимптотически нормальна
    $$\hat{\theta}_n\xrightarrow{n\to\infty}\mathcal{N}\left(\theta;\dfrac{\sigma^2\left(\theta\right)}{n}\right),$$
    тогда, чтобы построить доверительный интервал, нужно воспользоваться одним из двух подходов:
    \begin{enumerate}
        \item Нормировать величину $\theta$
        \item Найти функцию преобразования $g(u)$
    \end{enumerate}
    В нашем случае я буду использовать первый подход. Так как мы снова работаем с нормальным распределением (оценка, не $f_X(x)$), то рассуждения относительно
    построения доверительного интервала для параметра $\theta$ аналогичны. Обозначим $u_{1-\frac{\alpha}{2}}$ квантиль порядка $1-\frac{\alpha}{2}$
    $$\Bigg|\sqrt{n}\cdot\dfrac{\hat{\theta}_n-\theta}{\sigma\left(\theta\right)}\Bigg|\sim\mathcal{N}\left(0,1\right),\,\,\,\, \Bigg|\sqrt{n}\cdot\dfrac{\hat{\theta}_n-\theta}{\sigma\left(\theta\right)}\Bigg|\leq u_{1-\frac{\alpha}{2}}$$
    Раскроем модуль и выразим параметр $\theta$ аналогично первому заданию
    $$\hat{\theta}_n-\dfrac{\sigma\left(\theta\right)}{\sqrt{n}}u_{1-\frac{\alpha}{2}}\leq\theta\leq\hat{\theta}_n+\dfrac{\sigma\left(\theta\right)}{\sqrt{n}}u_{1-\frac{\alpha}{2}}$$
    Проведем небольшую замену для удобства
    $$\hat{\theta}_n-SE\left[\hat{\theta}_n\right]u_{1-\frac{\alpha}{2}}\leq\theta\leq\hat{\theta}_n+SE\left[\hat{\theta}_n\right]u_{1-\frac{\alpha}{2}},$$
    $$SE\left[\hat{\theta}_n\right]=\sqrt{\text{Var}\left[\hat{\theta}_n\right]},\text{ где SE -- Standard Error}$$
    В нашем случае оцениваем медиану. Заменим $\hat{\theta}_n$ на $\hat{m}$ и получим неравенство, к которому нужно будет привести наши данные, чтобы получить
    доверительный интервал для медианы
    $$\hat{m}-SE\left[\hat{m}\right]u_{1-\frac{\alpha}{2}}\leq m\leq\hat{m}+SE\left[\hat{m}\right]u_{1-\frac{\alpha}{2}}$$
    Так как напрямую найти оценку медианы мы не можем, то найдем оценку параметра $\lambda$, после чего найдем связь между $m$ и $\lambda$.
    Оценивать параметр $\lambda$ будем методом правдоподобия. Составим функцию правдоподобия -- произведение $f_X(x_i)$
    $$L\left(\lambda,x\right)=\prod_{i=1}^{n}f_X\left(x_i\right)=\lambda^n\cdot\exp{\left\{-\lambda\sum\limits_{i=1}^{n}x_i\right\}}$$
    Максимизируем функцию правдоподобия, чтобы найти $\hat{\lambda}$. Для этого возьмем частную производную по $\lambda$ от логарифма функции правдоподобия и приравняем к нулю
    $$\hat{\lambda}=\text{argmax}\left(L\left(\lambda,x\right)\right)\Rightarrow \dfrac{\partial\ln{L\left(\lambda,x\right)}}{\partial\lambda}=0$$
    Логарифмируем функцию правдоподобия
    $$\ln{L\left(\lambda,x\right)}=\ln{\left(\lambda^n\cdot\exp{\left\{-\lambda\sum\limits_{i=1}^{n}x_i\right\}}\right)}=
    \ln{\lambda^n}+\ln{\exp{\left\{-\lambda\sum\limits_{i=1}^{n}x_i\right\}}},$$
    $$\ln{\lambda^n}=n\cdot\ln{\lambda},\,\,\,\,\ln{\exp{\left\{-\lambda\sum\limits_{i=1}^{n}x_i\right\}}}=-\lambda\sum\limits_{i=1}^{n}x_i\cdot\ln{e}=-\lambda\sum\limits_{i=1}^{n}x_i,$$
    $$\ln{L\left(\lambda,x\right)}=n\ln{\lambda}-\lambda\sum\limits_{i=1}^{n}x_i$$
    Теперь возьмем производную и найдем оценку параметра $\lambda$
    $$\dfrac{\partial\ln{L\left(\lambda,x\right)}}{\partial\lambda}=\left(n\ln{\lambda}-\lambda\sum\limits_{i=1}^{n}x_i\right)_{\lambda}^{\prime}=
    \dfrac{n}{\lambda}-\sum\limits_{i=1}^{n}x_i=0\Rightarrow\dfrac{n}{\lambda}=\sum\limits_{i=1}^{n}x_i\Rightarrow\hat{\lambda}=\dfrac{n}{\sum_{i=1}^{n}x_i},$$
    $$\overline{X}=\dfrac{1}{n}\sum\limits_{i=1}^{n}x_i\Rightarrow\sum\limits_{i=1}^{n}x_i=n\cdot\overline{X}\Rightarrow\hat{\lambda}=\dfrac{n}{n\cdot\overline{X}}=\dfrac{1}{\overline{X}}$$
    Таким образом, мы нашли оценку параметра $\lambda$
    $$\hat{\lambda}=\dfrac{1}{\overline{X}}$$
    Выведем связь $m$ и $\lambda$ через функцию распределения, которая имеет вид
    $$F_X(x)=\begin{cases}
        1-\exp{\left\{-\lambda x\right\}},& x\geq0\\
        0,& x<0
    \end{cases}$$
    Функция распределения находится на промежутке от 0 до 1, следовательно медиана для некоторого $x=m$ находится в значении функции распределения, равного 0.5
    $$F_X(x)\in\left[0;1\right]\Rightarrow\text{Med}\left[x\right]=0.5$$
    Здесь мы и найдем связь $m$ и $\lambda$ и выразим $\hat{m}$ через $\hat{\lambda}$ (такое возможно при наличии функциональной связи между параметрами)
    $$F_X(x=m)=1-\exp{\left\{-\lambda m\right\}}=0.5\Rightarrow\exp{\left\{-\lambda m\right\}}=0.5\Rightarrow\ln{e^{-\lambda m}}=\ln{0.5},$$
    $$\ln{e^{-\lambda m}}=-\lambda m\cdot\ln{e}=-\lambda m,\,\,\,\,\ln{0.5}=\ln{2^{-1}}=-\ln{2},$$
    $$-\lambda m=-\ln{2}\Rightarrow\lambda m=\ln{2}\Rightarrow m=\dfrac{\ln{2}}{\lambda}\Rightarrow\hat{m}=\dfrac{\ln{2}}{\hat{\lambda}}=\ln{2}\cdot\overline{X}$$
    Таким образом, оценка медианы имеет вид
    $$\hat{m}=\overline{X}\ln{2}$$
    При этом, согласно ЦПТ (выборка из независимых одинаково распределенных $x_i$, матожидание и дисперсия конечны по wiki. константа не повлияет на распределение),
    $$\hat{m}=\overline{X}\ln{2}\sim\mathcal{N}\left(\dfrac{\ln{2}}{\lambda},\dfrac{\left(\ln{2}\right)^2}{n\lambda^2}\right),$$
    а значит оценка медианы имеет асимптотически нормальное распределение при больших $n$.
    Осталось вычислить дисперсию, по которой найдем Standard Error. Используем свойства дисперсии
    $$\text{Var}\left[\hat{m}\right]=\text{Var}\left[\,\overline{X}\ln{2}\right]=\text{Var}\left[\dfrac{\ln{2}}{n}\sum_{i=1}^{n}x_i\right]=\left(\dfrac{\ln{2}}{n}\right)^2\text{Var}\left[\sum_{i=1}^{n}x_i\right],$$
    $$\text{Var}\left[\sum_{i=1}^{n}x_i\right]=\text{/случ. вел. независимы/}=\sum_{i=1}^{n}\text{Var}\left[x_i\right]=n\cdot\text{Var}\left[x\right],$$
    $$\text{Var}\left[x\right]=\text{/wiki/}=\dfrac{1}{\lambda^2}\Rightarrow n\cdot\text{Var}\left[x\right]=\dfrac{n}{\lambda^2},$$
    $$\left(\dfrac{\ln{2}}{n}\right)^2\text{Var}\left[\sum_{i=1}^{n}x_i\right]=\left(\dfrac{\ln{2}}{n}\right)^2\cdot\dfrac{n}{\lambda^2}=\dfrac{\left(\ln{2}\right)^2}{\lambda^2 n}$$
    Таким образом, дисперсия оценки медианы имеет вид
    $$\text{Var}\left[\hat{m}\right]=\dfrac{\left(\ln{2}\right)^2}{\lambda^2 n}$$
    Вычислим стандартную ошибку
    $$SE\left[\hat{m}\right]=\sqrt{\text{Var}\left[\hat{m}\right]}=\sqrt{\dfrac{\left(\ln{2}\right)^2}{\lambda^2 n}}=\dfrac{\ln{2}}{\lambda\sqrt{n}}$$
    Подставим найденные оценку медианы и стандартную ошибку оценки медианы в искомый доверительный интервал, который мы выразили в начале задания.
    Таким образом, доверительный интервал для медианы будет иметь вид
    $$\overline{X}\ln{2}-\dfrac{\ln{2}}{\lambda\sqrt{n}}u_{1-\frac{\alpha}{2}}\leq m\leq\overline{X}\ln{2}+\dfrac{\ln{2}}{\lambda\sqrt{n}}u_{1-\frac{\alpha}{2}}$$


    Проведем эксперименты. Импортируем необходимые библиотеки
    \begin{lstlisting}[label=imps2, caption={Импорт необходимых библиотек}]
    import numpy as np
    import scipy.stats as st
    \end{lstlisting}
    Запишем в переменные известные данные. Медиану мы выразили при поиске ее оценки. Зададим уровень
    значимости $\alpha=0.05$
    \begin{lstlisting}[label=vars2, caption={Запись известных параметров в переменные}]
    _lambda = 1
    med = np.log(2) / _lambda
    n = 25
    alpha = 0.05
    it = 1000
    \end{lstlisting}
    Реализуем основной алгоритм. Написан аналогично первому заданию, однако теперь выборка одна, она
    имеет экспоненциальное распределение, и, доверительный интервал высчитывается по другой формуле
    \begin{lstlisting}[label=code2, caption={Реализация алгоритма для задания 2}]
    count = 0
    for i in range(it):
        X = np.random.exponential(scale = 1 / _lambda, size=n)

        X_mean = np.mean(X)
        hat_m = np.log(2) * X_mean

        sigma = np.log(2) / (_lambda * np.sqrt(n))

        z = st.norm.ppf(1 - alpha / 2)

        lower_bound = hat_m - z * sigma
        upper_bound = hat_m + z * sigma

        print(f'{lower_bound:.4f}<={med:.4f}<={upper_bound:.4f}')

        if lower_bound <= med <= upper_bound:
            count += 1

    print(f'covers_med_count={count}, ratio={count / it}')
    \end{lstlisting}


    Выведем доверительный интервал для $n=25,\,it=1$
    \begin{lstlisting}[label=int2, caption={Доверительный интервал для $n=25$}]
    0.2781<=0.6931<=0.8216
    \end{lstlisting}
    Посчитаем количество медиан, попавших в наш доверительный интервал при $n=25,\,it=1000$
    \begin{lstlisting}[label=n25_2, caption={Количество медиан внутри интервала и отношение при $n=25$}]
    covers_med_count=950, ratio=0.95
    \end{lstlisting}
    Проведем аналогичные действия для $n=10000$
    \begin{lstlisting}[label=n10000_2, caption={Количество медиан внутри интервала и отношение при $n=10000$}]
    covers_med_count=948, ratio=0.948
    \end{lstlisting}


    Можем наблюдать, что результаты аналогичны первому заданию, наши выводы подтвердились и тут.


    \section{Приложения}
    \subsection{Приложение 1}
    Доверительные интервалы для 1000 итераций для первого и второго заданий можно посмотреть в прикрепленном файле \texttt{intervals.txt}
    \attachfile{C:/Users/alexe/code/math-stats/lab_2/intervals.txt}
\end{document}