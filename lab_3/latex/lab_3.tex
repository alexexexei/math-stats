\documentclass[a4paper, 12pt]{article}
\usepackage[utf8x]{inputenc}
\usepackage{cmap}
\usepackage[english, russian]{babel}
\usepackage{indentfirst}
\usepackage[left=20mm, top=20mm, right=20mm, bottom=20mm]{geometry}
\usepackage{tikz}
\usepackage{float}
\usepackage{amsmath, amsfonts, amssymb}
\usepackage{graphicx}
\usepackage{fancybox, fancyhdr}
\usepackage{hyperref}
\usepackage{listings}
\usepackage{caption}
\usepackage{subcaption}
\usepackage{xcolor}
\pagestyle{fancy}
\fancyhf{}
\fancyhead[L]{Лабораторная работа №3}
\fancyhead[R]{Математическая статистика}
\fancyfoot[C]{\thepage}
\graphicspath{{images/}}
\usetikzlibrary{patterns}
\definecolor{LightGray}{gray}{0.95}
\definecolor{LightGray2}{gray}{0.7}
\lstdefinestyle{code}{
    language=Python, % replace with needed language
    basicstyle=\footnotesize\ttfamily,
    backgroundcolor=\color{LightGray},
    showspaces=false,
    showstringspaces=false,
    showtabs=false,
    tabsize=4,
    captionpos=b,
    breaklines=true,
    breakatwhitespace=false,
    frame=single,
    rulecolor=\color{LightGray2},
    linewidth=\linewidth,
    keywordstyle=\color{blue}\bfseries,
    commentstyle=\color{green!40!black},
    stringstyle=\color{purple},
    escapeinside={\%*}{*)},
    inputencoding=utf8x,
    xleftmargin=0pt,
    framexleftmargin=0pt,
    framexrightmargin=0pt
}
\lstset{style=code}
\hypersetup{
    colorlinks=true,
    linkcolor=blue,
    filecolor=magenta,
    urlcolor=cyan,
    pdftitle={contents setup},
    pdfpagemode=FullScreen,
}
\setlength{\parskip}{1.5mm}
\setlength{\headheight}{15pt}
\setlength{\footskip}{15pt}
\allowdisplaybreaks
\DeclareMathOperator{\sinc}{sinc}
\newcommand{\frc}[2]{\raisebox{2pt}{$#1$}\big/\raisebox{-3pt}{$#2$}}

\begin{document}
    \begin{titlepage}

        \begin{center}
        \includegraphics[width=0.3\textwidth]{itmo.png} % requires itmo.png in /images folder
        \vfill
        
        Федеральное государственное автономное образовательное учреждение высшего образования
        «Национальный Исследовательский Университет ИТМО»\\
        
        \vfill
        {\large\bf ЛАБОРАТОРНАЯ РАБОТА №3}\\
        {\large\bf ПРЕДМЕТ «МАТЕМАТИЧЕСКАЯ СТАТИСТИКА»}\\
        {\large\bf ТЕМА «СТАТИСТИЧЕСКИЕ ГИПОТЕЗЫ»}\\
        Вариант 5
        \vfill

        \begin{flushright}
            \begin{minipage}{.45\textwidth}
            {
                \hbox{Преподаватель: Лимар И. А.}
                \hbox{Студент: Румянцев А. А.}
                \hbox{Поток: Мат Стат 31.2}
                \hbox{}
                \hbox{Факультет: СУиР}
                \hbox{Группа: R3341}
            }
            \end{minipage}
        \end{flushright}
        
        \vfill
                
        Санкт-Петербург\\
        2024
        \end{center}
    \end{titlepage}
    
    \tableofcontents

    \newpage
    \section{Задание}
    Для каждой проблемы нужно провести два статистических теста, если не сказано иное, причем
    первый из критериев нужно реализовать самостоятельно (считать и выводить значение статистики,
    критическое значение, p-value), в качестве второго можно воспользоваться готовой реализацией. Также
    нужно отдельно указывать, как формализуются $H_{0}$ и $H_{1}$ для выбранных тестов. Уровень значимости
    выбираете сами.


    \subsection{Условие}
    В файле \href{https://drive.google.com/file/d/1KRbKtVb6Xkyc8_2gKT9G6N5N_yD7FQXC/view}{song\_data.csv} представлены данные
    о музыкальных произведениях
    \begin{enumerate}
        \item Предположите, с каким вероятностным законом распределен рейтинг песни. С помощью стат.
        теста подтвердите/опровергните это предположение.
        \item Верно ли, что распределение рейтинга у коротких и длинных песен одинаково (порог
        продолжительности выбирайте сами)?
        \item Верно ли, что рейтинг и танцевальность связаны?
    \end{enumerate}


    \subsection{Выполнение}
    тут будет выполнение
\end{document}